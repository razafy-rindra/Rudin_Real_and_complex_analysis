\documentclass[hidelinks]{article}
\usepackage[margin=0.5in]{geometry}
\usepackage[utf8]{inputenc}

\usepackage{amsmath}
\usepackage{amsthm}
\usepackage{amssymb}
\usepackage{enumerate}
\usepackage{chngcntr}
\usepackage{mathtools}
\usepackage{enumitem}
\usepackage{listings}
\usepackage{hyperref}
\usepackage[dvipsnames]{xcolor}
\usepackage[bb=boondox]{mathalfa}
\newcommand{\Z}{\mathbb{Z}}
\newcommand{\C}{\mathbb{C}}
\newcommand{\HH}{\mathbb{H}}
\newcommand{\Q}{\mathbb{Q}}
\newcommand{\R}{\mathbb{R}}
\newcommand{\N}{\mathbb{N}}
\newcommand{\F}{\mathbb{F}}
\newcommand{\qbinom}{\genfrac{[}{]}{0pt}{}}
\DeclarePairedDelimiter\ceil{\lceil}{\rceil}
\DeclarePairedDelimiter\floor{\lfloor}{\rfloor}

\usepackage[dvipsnames]{xcolor}

\newtheorem{theorem}{Theorem}
\newtheorem{corollary}{Corollary}[theorem] 
\newtheorem{lemma}[theorem]{Lemma} 
\newtheorem{proposition}{Proposition}
\newcommand{\greenparagraph}[1]{\textcolor{ForestGreen}{\textbf{#1}}}

\theoremstyle{definition}
\newtheorem{definition}{Definition}[section]
\theoremstyle{remark}
\newtheorem*{remark}{Remark}
\theoremstyle{remark}
\newtheorem*{note}{Note}
\theoremstyle{definition}
\newtheorem{example}{Example}[definition]
\newcounter{exercise}[subsection]
\newenvironment{exercise}{\refstepcounter{exercise}\textbf{Exercise~\theexercise}}{}
\counterwithin*{equation}{section}
\counterwithin*{equation}{subsection}
\lstset{
  basicstyle=\ttfamily,
  mathescape
}
\usepackage{makeidx}
\makeindex

\begin{document}
    %\begin{theindex}
    %   \item Derivatives of Measures
    %  \subitem symmetric derivative~\ref{sym-der}
    % \subitem maximal function\ref{max-fun}
    %\end{theindex}
    \section{Differentiation}
    \subsection{Derivatives of Measures}
    \begin{theorem}
        Suppose $\mu$ is a complex Borel measure on $\R^1$ and \begin{equation}
            f(x) = \mu((-\infty, x)) \text{ for }x\in \R^1
        \end{equation}

        If $x\in \R^1$ and $A$ is a complex number, TFAE\begin{enumerate}[label = (\alph*)]
            \item $f$ is differentiable at $x$ and $f'(x) = A$.
            \item For all $\epsilon>0$, there exists $\delta>0$ such that\begin{equation}
                |\frac{\mu(I)}{m(I)}-A|<\epsilon
            \end{equation}

            for every open segment $I$ that contains $x$ and whose length is less than $\delta$. Note $m$ is the Lebesgue measure on $\R^1$.
        \end{enumerate}

        % \begin{proof}
        %     $(a)\Rightarrow (b)$         
        %     Since $f'(x) = A$, we have, for all $\epsilon>0$ there is a $\delta>0$ such that for $(t,x)$ with $|t-x|<\delta$:\begin{align*}
        %         |\frac{f(t)-f(x)}{t-x} - f'(x)| &= |\frac{\mu([t,x))}{t-x} - A| = |\frac{\mu([t,x))}{m([t,x))} - A| <\epsilon \tag{$\dagger$}
        %     \end{align*}

        %     So for $I = (a,b)$ is any open interval containg $x$, of length less than $\delta$. Now let $\{t_n\}$ be such that $a<\ldots<t_n<t_{n-1}<\ldots < t_1$.
        % \end{proof}
    \end{theorem}

    \begin{definition}
        Let us fix a dimension $k$, denote the open ball with center $x\in \R^k$ and radius $r>0$ by \[B(x,r) = \{y\in \R^k \colon |y-x|<r\}\]
        We associate to any Borel measure $\mu$ on $\R^k$ the quotients:\[({Q_r}\mu)(x) = \frac{\mu(B(x,r))}{m(B(x,r))}\]
        Where $m$ is the Lebesgue measure on $R^k$.

        \

        We define the \textbf{symmetric derivative}\index{symmetric derivative} to be \[(D\mu)(x) = \lim_{r\rightarrow 0}({Q_r}\mu)(x)\]
    \end{definition}

    \begin{definition}
        Using the same notation as above we define the \textbf{maximal function}\index{maximal function} $M\mu$, for $\mu\geq 0$, to be defined by \[(M\mu)(x) = \sup_{0<r<\infty}(Q_r\mu)(x)\]
    \begin{remark}
        The maximal function of a complex Borel measure $\mu$ is, by definition, its total variation $|\mu|$.
    \end{remark}
    \end{definition}
    \begin{proposition}
        The functions $M\mu\colon R^k\rightarrow [0,\infty]$ are lower semicontinuous, hence measurable.
        \begin{proof}
            Assume $\mu\geq 0$, and let $\lambda>0$ and $E = \{M\mu > \lambda\}$. Fix $x\in E$. Then there is an $r>0$ such that:\[\mu(B(x,r)) = tm(B(x,r)) \text{ for some }t>\lambda\]

            Indeed since $\sup_{0<r<\infty}\frac{\mu(B(x,r))}{m(B(x,r))} >\lambda$. So for some $r$, we have $\frac{\mu(B(x,r))}{m(B(x,r))} >\lambda$. Letting $t = \frac{\mu(B(x,r))}{m(B(x,r))}$ gives us the desired result. 
    
            Furthermore there is a $\delta>0$ such that:\[(r+\delta)^k<\frac{{r^k}t}{\lambda}\]
        If $|y-x|<\delta$, then $B(y,r+\delta) \supseteq B(x,r)$. Therefore
        \[
            \mu(B(y,r+\delta)) \geq \mu(B(x,r)) = tm(B(x,r)) = t[\frac{r}{{(r+\delta)}^k}m(B(y,r+\delta)) > \lambda m(B(y,r+\delta))]   
        \]
        Thus $B(x,\delta)\subseteq E$. So $E$ is open.
        \end{proof}
    \end{proposition}

    \begin{lemma}\label{7.3}
        If $W$ is the union of a finite collection of balls $B(x_i,r_i)$, with $i\leq i \leq N$. Then there is a set $S\subseteq \{1,\ldots,N\}$ so that:\begin{enumerate}[label = (\alph*)]
            \item the balls $B(x_i,r_i)$ with $i\in S$ are disjoint,
            \item \[W\subseteq \bigcup_{i\in S}B(x_i,3r_i),\]
            \item \[m(W)\leq 3^k\sum_{i\in S}m(B(x_i,r)i).\]
        \end{enumerate}

        \begin{proof}
            Order the balls $B_i = B(x_i,r_i)$ such that $r_1\geq r_2\geq \cdots \geq r_N$. Put $i_1 = 1$, discard all the $B_j$ that intersect with $B_{i_1}$. Let $B_{i_2}$ the first of our remaining balls, and discard all $B_j$ with $j>i_2$ that intersect $B_{i_2}$, and let $B_{i_3}$ be the first of the remaining ones, etc\dots

            This process stops after a finite number of steps, since we only have a finite collection of balls, and we let $S = \{i_1,i_2,\ldots\}$.
        $(a)$ holds by definition and $(c)$ follows from $(b)$ since $m(B(x_i,3r_i)) = {3^k}m(B(x_i,r_i))$.

        So we just need to show $(b)$. But notice for every discarded $B_j$, $B_j\cap B_i\neq \emptyset$ for some $i\in S$, where $r_i>r_j$. Assume that $X\in B_j\cap B_i$. We see that for all $x\in B_j$ we have:\begin{align*}
            |x-x_i| &\leq |x-X| + |X-x_i|\\
                    &\leq |x-x_j| + |x_j-X| + |X-x_i|\\
                    &< r_j + r_j + r_i \text{ since }x,X\in B_j\text{ and }X\in B_i\\
                    &< 3r_i \text{ since }r_j\leq r_i
        \end{align*}
        So we see that $B_j\subseteq B(x_i,3r_i)$. This gives us $(b)$.
        \end{proof}
    \end{lemma}

    \paragraph*{The maximal theorem}\begin{theorem}\label{maximal}
        If $\mu$ is a complex Borel measure on $\R^k$ and $\lambda$ is a positive number, then\[m\{M\mu > \lambda\}\leq 3^k\lambda^{-1}||\mu|| \tag{i} \]
        Here $||\mu|| = |\mu|(\R^k)$ and $m\{M\mu > \lambda\}$ is an abbreviation of $m(\{x\in \R^k\colon (M\mu)(x) > \lambda\})$
    \begin{proof}
        Fix $\mu$ and $\lambda$. Let $K$ be a compact subset of the open set $\{M\mu >\lambda\}$. Each $x\in K$ is the center of an open ball $B$ for which 
        \[|\mu|(B)>\lambda m(B)\]

        Some finite collection of these $B$'s covers $K$ and Lemma~\ref{7.3} tells us there is a disjoint subcollection $\{B_1,\ldots,B_n\}$ such that:\begin{equation*}
            m(K)\leq 3^k\sum_{1}^n m(B_i)\leq 3^k\lambda^{-1}\sum_{1}^n|\mu|(B_i) \leq 3^l\lambda^{-1}||\mu||
        \end{equation*}
        The disjointess of the $B_i$'s was used in the last inequality. So (i) follows by taking the supremum over all compact $K\subseteq \{M\mu > \lambda\}$.
    \end{proof}
    \end{theorem}
    \paragraph*{Weak $L^1$}
    If $f\in L^1(\R^k)$ and $\lambda >0$, then \[m\{|f|>\lambda\}\leq \lambda^{-1}||f||_1\]
    because, if we let $E = \{|f|>\lambda\}$, we have:\begin{equation*}
        \lambda m(E)\leq \int_R |f|dm\leq \int_{\R^k}|f| dm = ||f||_1
    \end{equation*}
    \begin{definition}
        Any measurable function $f$ for which:\[\lambda m\{|f|>\lambda\}\]
        is a bounnded funtion of $\lambda$ on $(0,\infty)$ is said to belong to \textbf{weak}\index{weak} $L^1$\index{weak $L^1$}
    \end{definition}
    So from above we see that the weak $L^1$ contains $L^1$. But it is also larger since for example if we let $f = \frac{1}{x}$ on $(0,1)$, then for any $\lambda>0$, we have
    \[
        \frac{1}{x}>\lambda\iff x<\frac{1}{\lambda}    
    \]

    So we have $\lambda\cdot m\{|f|>\lambda\} \leq \lambda\cdot m(0,\frac{1}{\lambda}) = 1<\infty$. So $\frac{1}{x}$ is weak $L^1$.
    \begin{definition}
        We associate to each $f\in L^1(\R^k)$ its \textbf{maximal function}\index{maximal function} $Mf\colon \R^k\rightarrow [0,\infty]$ by setting\begin{equation*}
            (Mf)(x) = \sup_{0<r<\infty}\frac{1}{m(B_r)}\int_{B(x,r)}|f|~dm
        \end{equation*}
    \end{definition}
    If we identify $f$ with the measure $\mu$ given by $d\mu = f~dm$, we see that this defintion agrees with the previously defined $M\mu$. So theorem~\ref{maximal} states that the ``maximal operator'' $M$ sends $L^1$ to weak $L^1$, witha bound (namely $3^k$)
    that depends only on the space $\R^k$, i.e: For every $f\in L^1(\R^k)$ and every $\lambda>0$\begin{equation*}
        m\{Mf>\lambda\} \leq 3^k\lambda^{-1}||f||_1
    \end{equation*}

    \paragraph*{Lebesgue points} 
    \begin{definition}
        If $f\in L^1(\R^k)$, any $x\in \R^k$ for which it is true that\[\lim_{r\rightarrow 0}\frac{1}{m(B_r)}\int_{B(x,r)} |f(y)-f(x)|~dm(y) = 0\]
        is called a \textbf{Lebesgue point}\index{Lebesgue point} of $f$.
    \end{definition}
    For example this equation holds if $f$ is continuous at the point $x$. More generally this equation holds, if the averages of $|f-f(x)|$ are not too small on the balls centered at $x$, i.e. The Lebesgue points of $f$ are thepoints where $f$ doesn't oscillate too much.

    \begin{theorem}
        If $f\in L^1(R^k)$, then almost every $x\in \R^k$ is a Lebesgue point of $f$.
    \begin{proof}
        Let \[(T,f)(x) = \frac{1}{m(B_r)}\int_{B(x,r)} |f-f(x)|dm \text{ for }x\in \R^k, r>0\]

        Put \[(Tf)(x) = \limsup_{r\rightarrow 0}(T_r f)(x)\]

        Pick $y>0$, let $n\in \N^\ast$. By a theorem from chap $3$, there exists $g\in C(\R^k)$ so that $||f-g||_1 < \frac{1}{n}$. Let $h = f-g$.

        Since $g$ is continuous, $Tg = 0$, and since:\begin{align*}
            (T_r h)(x)&=\frac{1}{B_r}\int_{B(x,r)} |h-h(x)|dm\\
            &\leq \frac{1}{B_r}\int_{B(x,r)} (|h|+|h(x)|)dm\\
            &=(\frac{1}{B_r}\int_{B(x,r)} |h|dm)+|h(x)|
        \end{align*}
        So we have:\[Th\leq Mh+|h|\]
        But since $T_r f\leq T_r g+T_r h$ it follows that\[Tf\leq Mh+|h|\]
        Therefore \[\{Tf>2y\} \subseteq \underbrace{\{Mh>y\}\cup\{|h|>y\}}_{E(y,n)}\]

        Since $||h||_1<\frac{1}{n}$, by theorem~\ref{maximal} we can see that\[m(E(y,n))\leq \frac{3^k+1}{yn}\]

        Note $\{Tf>2y\}$ is independant of $n$. Hence\[\{Tf>2y\} \subseteq \bigcap_{n=1}^\infty E(y,n)\]

        This intersection has measure zero, so $\{Tf>2y\}$ is a subset of a set of measure zero. 
        So since Lebesgue measure is complete $\{Tf>2y\}$ is measurable and has measure zero. This is true for all $y>0$ so $Tf = 0$ a.e.

        So note if $(Tf)(x) = \limsup_{r\rightarrow 0} (T_r f)(x) = 0$, then since $(T_r f)(x)\geq 0$ we see that this means that $0\leq \liminf (T_r f)(x) \leq \limsup (T_r f)(x) =0$.
        
        So we have $\lim_{r\rightarrow 0} (T_r f)$ exists and is equal to zero, so $x$ is a Lebesgue point. So almost every point $x\in \R^k$ is a Lebesgue point of $f$. 
    \end{proof}
    \end{theorem}

    \begin{definition}
        Recall that by the Radon-Nikodym theorem if $\mu$ is a positive $\sigma$-finite measure on a $\sigma$-algebra $\mathcal{M}$ in a set $X$, and $\lambda$ is a complex measure on $\mathcal{M}$ such that $\lambda\ll \mu$:\[\lambda(E) = \int_E f d\mu \] For some $f\in L^1(\mu)$

        $f$ is called the \textbf{Radon-Nikodym derivative}of $\mu$ with respect to $m$\index{Radon-Nikodym derivative} and is denoted \[f = \frac{d\lambda}{d\mu}\]
    \end{definition}

    \begin{theorem}\label{derivative}
        Suppose $\mu$ is a complex Borel measure on $\R^k$, and $\mu \ll m$. Let $f$ be the Radon-Nikodym derivative of $\mu$ with respect to $m$. Then $D\mu = f$ a.e. $[m]$, and
        \[\mu(E) = \int_E(D\mu)~dm\] for all Borel sets $E\subseteq \R^k$.
        \begin{proof}
            \[\mu(E) = \int_E f~dm\] For all Borel sets $E\subseteq \R^k$.

            Let $x$ be a Lebesgue point and $\Gamma_r = \frac{1}{B_r}\int_{B(x,r)}f~dm$. Then we have:\begin{equation*}
                0\leq |\Gamma_r - f(x)| = |\frac{1}{B_r}\int_{B(x,r)}(f-f(x))~dm| \leq \frac{1}{B_r}\int_{B(x,r)}|f-f(x)|~dm
            \end{equation*}

            Taking limits we see that \[f(x) = \lim_{r\rightarrow 0}\frac{1}{m(B_r)}\underbrace{\int_{B(x,r)}f~dm}_{\mu(B(x,r))} = \lim_{r\rightarrow 0}\frac{\mu(B(x,r))}{m(B_r)} = (D\mu)(x) \]
        
            Thus $(D\mu)(x)$ exists and equals to $f(x)$ at every Lebesgue point of $f$, so a.e.
        \end{proof}
    \end{theorem}

    \paragraph*{Nicely shirinking sets}
    \begin{definition}
        Suppose $x\in \R^k$. A sequence $\{E_i\}$ of Borel sets in $\R^k$ is said to \textbf{shrink to }$x$\textbf{ nicely}\index{shrink to $x$ nicely} if there is a number $\alpha>0$ with the following property:\begin{center}
            There is a sequence of balls $B(x,r_i)$ with $\lim r_i = 0$, such that $E_i\subseteq B(x,r_i)$ and:\[m(E_i) \geq \alpha m(B(x,r_i)) \ \text{ for }i=1,2,3,\ldots \]
        \end{center}
    \end{definition}

    \begin{theorem}
        Associate to each $x\in \R^k$ a sequence $\{E_i(x)\}$ that shrinks to $x$ nicely, and let $f\in L^1(\R^k)$. Then \[f(x) = \lim_{i\rightarrow \infty}\frac{1}{m(E_i(x))}\int_{E_i(x)} f~dm \]
        At every Lebesgue point of $f$.

        \begin{proof}
            Let $x$ be a Lebesgue point of $f$ and let $\alpha(x)$ and $B(x,r_i)$ be the positive number and the balls associate with $\{E_i(x)\}$. Since $E_i(x)\subseteq B(x,r_i)$ we have:
            \[\int_{E_i(x)}|f-f(x)|~dm \leq \int_{B(x,r_i)}|f-f(x)|~dm \]

            Furthermore, $\alpha m(B(x,r_i))\leq m(E_i) \iff \frac{\alpha}{m(E_i)}\leq \frac{1}{m(B(x,r_i))}$. Putting this all together we get:

            \[\frac{\alpha}{m(E_i)}\int_{E_i(x)}|f-f(x)|~dm \leq \frac{1}{m(B(x,r_i))}\int_{B(x,r_i)}|f-f(x)|~dm\]

            Since $x$ is a Lebesgue point RHS converges to $0$, so the LHS also converges to zero by squeeze.
        \end{proof}
    \end{theorem}

    \begin{corollary}\label{7.11}
        If $f\in L^1(\R^1)$ and \[F(x) = \int_{-\infty}^x f~dm, \text{ for }x\in \R\]
        then $F'(x) = f(x)$ at every Lebesgue point of $f$.

        \begin{proof}
            Let $x$ be a Lebesgue point, and $\{\delta_i\}$ be a sequence of positive numbers that converges to $0$. Letting $E_i(x) = [x,x+\delta_i]$, the previous theorem tells us that 
            \[f(x) = \lim_{i\rightarrow\infty} \frac{1}{\delta_i}\int_{x}^{x+\delta_i}f~dm = \lim_{i\rightarrow\infty}\frac{1}{\delta_i}(\int_{-\infty}^{x+\delta_i} f~dm - \int_{-\infty}^x f~dm) = \lim_{i\rightarrow\infty}\frac{F(x+\delta_i) - F(x)}{\delta_i} \]

            Since $\{\delta_i\}$ is any sequence of positive numbers converging to zero we have:\[f(x) = \lim_{h\rightarrow 0^+}\frac{F(x+h) - F(x)}{h}\]
            
            Likewise letting $G_i(x) = [x-\delta_i,x]$ we get \[f(x) = \lim_{i\rightarrow\infty}\frac{F(x-\delta_i) - F(x)}{\delta_i} = \lim_{h\rightarrow 0^-}\frac{F(x+h) - F(x)}{h} \]
        

            So we have:\[f(x) = \lim_{h\rightarrow 0}\frac{F(x+h) - F(x)}{h} = F'(x) \text{ at every Lebesgue point of }f\]
        \end{proof}
    \end{corollary}

    \paragraph*{Metric density}
    \begin{definition}
        Let $E$ be a Lebesgue measurable subset of $\R^k$. The \textbf{metric density}\index{metric density} of $E$ at a point $x\in \R^k$ is defined to be 
        \[
            \lim_{r\rightarrow 0}\frac{m(E\cap B(x,r))}{m(B(x,r))}    \text{ if this limit exists.}
        \]

        If we let $f$ be the characteristic function of $E$, and apply Theorem~\ref{derivative}, we see that the metric density of $E$ is $1$ at almost every point of $E$ and is $0$ at almost every point of $E^c$.

        Indeed let $x$ be a Lebesgue point if $\mu(B(x,r)) = \int_{B(x,r)} f~dm = m(E\cap B(x,r))$, it is clear that $\mu\ll m$ and so we have:\[
            \lim_{r\rightarrow 0}\frac{m(E\cap B(x,r))}{m(B(x,r))} = \lim_{r\rightarrow 0} \frac{\mu(B(x,r))}{m(B(x,r))} = (D\mu)(x) = f(x) = \begin{cases}
                1 \text{ if }x\in E\\
                0 \text{ if }x\not\in E
            \end{cases}
        \]
    \end{definition}
    \begin{corollary}
        If $\epsilon>0$, there is no set $E\subseteq \R^1$ such that\[\epsilon < \frac{m(E\cap I)}{m(I)}<1-\epsilon\] For every segment $I$.
        \begin{proof}
            Let $\epsilon>0$ assume that that such a $E\subseteq \R^1$ exists. Let $x$ be a Lebesgue point, from what we have seen in the definition of metric density we know that there is a $R$ such that:\begin{align*}
                |\frac{m(E\cap (x-R,x+R))}{m(x-R,x+R)}-0| &< \epsilon \text{ if }x\not\in E\\
                |\frac{m(E\cap (x-R,x+R))}{m(x-R,x+R)}-1| &< \epsilon \text{ if }x\in E
            \end{align*}         
            I.e. \begin{align*}
                \frac{m(E\cap (x-R,x+R))}{m(x-R,x+R)} < \epsilon \text{ or } 1-\epsilon<\frac{m(E\cap (x-R,x+R))}{m(x-R,x+R)}
            \end{align*}
        \end{proof}
    \end{corollary}

    We now look at differentiation of measures that are singular wrt $m$.
    \begin{theorem}\label{singular}
        Associate to each $x\in \R^k$ a sequence $\{E_i(x)\}$ that shrinks to $x$ nicely. If $\mu$ is a complex Borel measure and $\mu \perp m$, then\[
            \lim_{i\rightarrow \infty}\frac{\mu(E_i(x))}{m(E_i(x))} = 0 \text{ a.e. }[m]    
        \]

        \begin{proof}
            By the Jordan decomp theorem we just need to show that this is true with $\mu\geq 0$. In that case as we have seen in previous theorems:\[
                \frac{\alpha(x)\mu(E_i(x))}{m(E_i(x))} \leq \frac{\mu(E_i(x))}{m(B(x,r_i))} \leq \frac{\mu(B(x,r_i))}{m(B(x,r_i))} 
            \] 

            So if we can show that $(D\mu)(x) = 0$ a.e. [m], we will prove the result by taking limits in the above inequality.

            \

            The upper derivative $\bar{D}\mu$ is defined by:\[
                (\bar{D}\mu)(x) = \lim_{n\rightarrow \infty}\bigg[\sup_{0<r<1/n}(Q_r\mu)(x)\bigg] \text{ for }x\in \R^k
            \]
            Is a Borel function. 

            \

            Choose $\lambda>0$ and $\epsilon>0$. Since $\mu \perp m$, $\mu$ is concentrated on a set of Lebesgue measure $0$. The regularity of $\mu$ shows that there is a compact set $K$m with $m(K) = 0$, and $\mu(K)>||\mu||-\epsilon$.

            Define $\mu_1(E) = \mu(K\cap E)$, for any Borel set $E\subseteq \R^k$, and put $\mu_2 = \mu-\mu_1$. Then $||\mu_2||<\epsilon$, and for every $x$ outside $K$,\[
                (\bar{D}(\mu))(x) = (\bar{D}(\mu_2))(x) \leq (M\mu_2)(x).    
            \]
            Hence \[
                \{\bar{D}\mu>\lambda\}\subseteq K\cup \{M\mu_2>\lambda\},    
            \]
            And \[
                m\{\bar{D}\mu>\lambda\} \leq 3^k\lambda^{-1}||\mu_2||<3^k\lambda^{-1}\epsilon    
            \]

            Since this holds for all $\epsilon>0$ and $\lambda>0$, we find that $\bar{D}\mu = 0$ a.e. [m], so \[(D\mu)(x) = 0 \ a.e.[m]\]
            Which gives us our result.
        \end{proof}
    \end{theorem}

    \begin{corollary}
        Suppose that to each $x\in \R^k$ is associated to some sequence $\{E_i(x)\}$ that shrinks to $x$ nicely, and that $\mu$ is a complex Borel measure on $\R^k$.
    Let $d\mu = f~dm+d\mu_s$ be the Lebesgue decomposition of $\mu$ wrt $m$. Then\[
            \lim_{i\rightarrow \infty}\frac{\mu(E_i(x))}{m(E_i(x))} = f(x) \ a.e. [m] 
    \]
    In particular, $\mu\perp m$ if and only if $(D\mu)(x) = 0$ a.e. [m]
        \begin{proof}
            Let $\mu_a(E) = \int_E f~dm$, then recall that $\mu = \mu_a+\mu_s$, and \[\begin{cases}
                \mu_a\ll m\\
                \mu_s\perp m
            \end{cases}\]

            Then from theorem~\ref{derivative}
            \[
                \lim_{i\rightarrow \infty} \frac{\mu_a(E_i(x))}{m(E_i(x))} = f(x) \ a.e. [m]
            \]

            On the other hand from theorem~\ref{singular}
            \[
                \lim_{i\rightarrow \infty} \frac{\mu_s(E_i(x))}{m(E_i(x))} = 0 \ a.e. [m]
            \]

            So we have\[
                \lim_{i\rightarrow \infty}\frac{\mu(E_i(x))}{m(E_i(x))} = \lim_{i\rightarrow \infty}\frac{\mu_a(E_i(x))+\mu_s(E_i(x))}{m(E_i(x))} = \lim_{i\rightarrow \infty} \frac{\mu_a(E_i(x))}{m(E_i(x))} + \lim_{i\rightarrow \infty} \frac{\mu_s(E_i(x))}{m(E_i(x))} = f(x) \ a.e. [m]      
            \]

            (Indeed if we let $A,B$ be the sets where the first two equations fail, then our result fails on $A\cup B$, which is the union of two measure zero sets, and so is a measure zero set, so this equality is true almost everywhere).
        \end{proof}
    \end{corollary}

    \begin{theorem}
        If $\mu$ is a positive Borel measure on $\R^k$ and $\mu\perp m$, then\[
            (D\mu)(x) = \infty \ a.e. [\mu]    \tag{$\dagger$}
        \]

        \begin{proof}
            There is a Borel set $S\subseteq \R^k$ with $m(S) = 0$ and $\mu(\R^k\setminus S) = 0$, and there are open sets $V_j\supseteq S$ with $m(V_j)<\frac{1}{j}$, for $j=1,2,3,\ldots$

            For $N=1,2,3,\ldots$, let $E_N$ be the set of all $x\in S$ to which correspond radii $r_i = r_i(x)$, with $\lim r_i = 0$ such that \[
                \mu(B(x,r_i)) <  Nm(B(x,r_i)).  \tag{$\dagger\dagger$}
            \]
            Then $(\dagger)$ holds for all $s\in S\setminus \bigcup_N E_N$.

            \

            Fix $N$ and $j$, for the moment. Every $x\in E_N$ is in the center of a ball $B_x\subseteq V_j$, that satisfies ($\dagger\dagger$). Let $\beta_x$ be the open ball with center $x$ whose radius is $\frac{1}{3}$ of that of $B_x$. The union of the $\beta_x$ is an open set $W_{j,N}$ such that $E_N\subseteq W_{j,N}\subseteq V_j$
    
            \

            Let $K\subseteq W_{j,N}$ be compact. Finitely many $\beta_x$ cover $K$. Lemma~\ref*{7.3} shows that there is a finite set $F\subseteq E_N$ such that:\begin{enumerate}[label = (\alph*)]
                \item $\{\beta_x\colon x\in F\}$ is a disjoint collection, and
                \item $K\subseteq \bigcup_{x\in F} B_x$
            \end{enumerate}

            Therefore \begin{align*}
                \mu(K) &\leq \sum_{x\in F}\mu(B_x)\\
                    &< N\sum_{x\in F} m(B_x)\\
                    &= 3^k N\sum_{x\in F}m(\beta_x)\\
                    &\leq 3^k Nm(V_j)\\
                    &<3^k N/j
            \end{align*}
            This is true for any compact subset of $W_{j,N}$, since $W_{j,N}$ is open furthermore $\mu$ is a positive Borel measure on $\R^k$, so it is regular, therefore we have:
            
            \[
                \mu(W_{j,N}) = \sup\{\mu(K) \colon K\subseteq W_{j,N}\text{ is compact }\} < 3^k N/j   
            \]

            Now let $\Omega_N = \bigcap_j W_{j,N}$, then $E_N\subseteq \Omega_N$, and $\Omega_N$ is a $G_\delta$ (so is measurable), and $\mu(\Omega_N) = 0$, and so:\[(D\mu)(x) = \infty \text{ for all }x\in S\setminus\bigcup_N\Omega_N\]
            
            Since $\bigcup_N\Omega_N$ is a set of measure zero, we have the desired result.
        \end{proof}
    \end{theorem}
    \subsection{The Fundamental Theorem of Calculus}
    \paragraph*{Problems with the FTC when extending to the Lebesgue integral}
    \begin{enumerate}[label = (\alph*)]
        \item Let \[
            f(x) = \begin{cases}
                x^2\sin(x^{-2}) \text{ if }x\neq 0\\
                0 \text{ otherwise}
            \end{cases}    
        \]
        Then $f$ is differentiable at every point, but\[
            \int_0^1 |f'(t)|~dt =\infty    
        \]

        So $f'\not\in L^1$. But we still have:\[
            f(x) = \lim_{\epsilon\rightarrow 0}\int_\epsilon^x f'(t)~dt = \int_0^x f'(t)~dt    
        \]

        \item Suppose $f$ is continuous on $[a,b]$, $f$ is differentiable at almost every point of $[a,b]$ and $f'\in L^1$ on $[a,b]$. Do these assumptions imply that $f(x) - f(a) = \int_a^x f'$
        
        \textsc{NO!}

        \

        Choose $\{\delta_n\}$ so that $1 = \delta_0>\delta_1>\cdots$, where $\delta_n\rightarrow 0$, we define the sets $E_n$ recursively. Put $E_0 = [0,1]$ and if $n\geq 0$ and $E_n$ is constructed so that it is the union of $2^n$ disjoint closed intervals, each of length $2^{-n}\delta_n$. 
        
        Delete a segment in the center of each of the $2^n$ intervals, so that each $2^{n+1}$ intervals have length $2^{-(n+1)}\delta_{n+1}$, and let $E_{n+1}$ be the union of these $2^{n+1}$ intervals. So we have $E_1\supseteq E_2\supseteq \ldots$, and $m(E_n) = \delta_n$ for all $n$. Now let\[
            E = \bigcap_{n=1}^\infty E_n    
        \]

        Note since each $E_n$ is the finite union of closed sets, they are closed so $E$ is also closed, it is also bounded since it is contained in $[0,1]$. So $E$ is compact and $m(E) = \lim_{n\rightarrow \infty}m(E_n) = \lim_{n\rightarrow \infty}\delta_n = 0$. Put \[
            g_n = \delta_n^{-1}\chi_{E_n} \text{ and } f_n(x) = \int_0^x g_n(t)~dt \text{ for }n=0,1,2,\ldots    
        \]

        So note $f_n(0) = 0$ and $f_n(1) = \delta_n^{-1}\int_0^1 \chi_{E_n} = \delta_n^{-1}m(E_n) = 1$. Each $f_n$ is a monotonic function which is constant on each segment in $E_n^c$.
        If $I$ is one of the $2^n$ interval whose union is $E_n$, then\[
            \int_I g_n(t)~dt = \int_I g_{n+1}(t)~dt = 2^{-n}.    
        \]

        Therefore we see that \[
            f_{n+1}(x) = f_n(x)  \text{ for }x\not\in E_n   
        \]

        Now note that \[
            |f_n(x) - f_{n+1}(x)|\leq \int_I |g_n-g_{n+1}|<2^{-(n-1)} \text{ for } x\in E_n    
        \]

        So $\{f_n\}$ converges uniformely to a continuous monotonic function $f$, with $f(0) = 0$, $f(1) = 1$, and $f'(x) = 0$ for all $x\not\in E$. Since $m(E) = 0$, we see that $f' = 0$ almost everywhere. So\[
            f(x) \neq \int_0^x f' \text{ in general}    
        \]
    \end{enumerate}

    \begin{definition}\label{7.17}
            \begin{remark}
                Now we see that if $f'\in L^1$ and that \[
                    f(x) - f(a) = \int_a^x f'   \tag{FTC}
                \]
                Then there is a measure $\mu$ defined by $d\mu = f'~dm$. Since $\mu \ll m$, we know that there corresponds to each $\epsilon>0$ a $\delta>0$ such that $|\mu|(E)<\epsilon$, whenever $E$ is a union of disjoint segments whose total length is less than $\delta$. Since $f(y) - f(x) = \mu((x,y))$ if $a\leq x<y\leq b$, it follows that the next definition is necessary for (FTC). 
            \end{remark}\index{absolutely continuous}
            A complex function $f$, defined on an interval $I = [a,b]$ is said to be \textbf{absolutely continuous} on $I$ (or $f$ is AC on $I$) if for all $\epsilon>0$ there is a $\delta>0$ such that\[
                \sum_{i=1}^n|f(\beta_i)-f(\alpha_i)|<\epsilon    
            \]
            For all $n$m and any disjoint collection of segments $(\alpha_1,\beta_1),\ldots,(\alpha_n,\beta_n)$ in $I$ whose length satisfy\[
                \sum_{i=1}^n (\beta_i-\alpha_i)<\delta    
            \]
            \begin{remark}
                Such an $f$ is continuous since we can choose $n=1$.
            \end{remark}
    \end{definition}

    \begin{theorem}\label{7.18}
            Let $I=[a,b]$ and $f\colon I\rightarrow \R$ be continuous and nondecreasing. TFAE\begin{enumerate}[label=(\alph*)]
                \item $f$ is AC on $I$
                \item $f$ maps sets of measure $0$ to sets of measure of $0$.
                \item $f$ is differentiable a.e. on $I$, $f'\in L^1$ and \[
                    f(x) - f(a) = \int_a^x f'(t)~dt \text{ for }(a\leq x\leq b)    
                \]
            \end{enumerate}

            \begin{proof}
                \begin{itemize}
                    \item $(a)\Rightarrow (b)$
                    
                    Let $\mathcal{M}$ denote the $\sigma$-algebra of all Lebesgue measurable subsets of $\R$. Assume $f$ is AC on $I$, pick $E\subseteq I$ so that $E\in \mathcal{M}$ and $m(E) = 0$. We will show that $f(E)\in \mathcal{M}$ and $m(f(E)) = 0$. WLOG we assume that $E\subseteq (a,b)$.
                    Choose $\epsilon>0 $ and let $\delta>0$ be as in definition~\ref{7.17}. There is an open set $V$ with $m(V)<\delta$, so that $E\subseteq V\subseteq I$. Let $(\alpha_i,\beta)$ be the disjoint segment whose union is $V$, then $\sum (\beta_i-\alpha_i)<\delta$ and so \[
                        \sum |f(\beta_i)-f(\alpha_i)|<\epsilon \tag{$dagger$}
                    \]
                    We know that this holds for every partial sum of this series, so it holds for the whole series, even if $\dagger$ is an infite sum.
                    
                    Since $E\subseteq V$, $f(E)\subseteq \bigcup [f(\alpha_i),f(\beta_i)]$. The Lebesgue measure of this union is $\sum |f(\beta_i)-f(\alpha_i)|<\epsilon$. So $f(E)$ is a subset of a borel set of arbitrarily small measure. Since Lebesgue measure is complete, we see that $f(E)\in \mathcal{M}$ and $m(f(E)) = 0$.
                    \item $(b)\Rightarrow (c)$
                    
                    Define \[
                        g(x) = x+f(x) \text{ for }(a\leq x\leq b).    
                    \]
                    So note that if the $f$-image of some segment of length $\eta$ has length $\eta'$, then the $g$-image of this segment has length $\eta+\eta'$ (Indeed $m([x+f(x),y+f(y)]) = x-y+f(x)-f(y) = \eta+\eta'$).  
                    So we see that $g$ satisfies condition $(b)$. Now suppose $E\subseteq I$, $E\in \mathcal{M}$. Then $E = E_1\cup E_0$ where $m(E_0) = 0$ and $E_1$ is $F_\sigma$, by a previous theorem. Thus $E_1$ is a countable union of compact set and so is $g(E_1)$ since $g$ is continuous. Since $m(g(E_0)) = 0$, we have $g(E) = g(E_1)\cup g(E_0)$ so we conclude that $g(E)\in \mathcal{M}$.

                    \

                    Therefore we can define \[
                        \mu(E) = m(g(E)) \text{ for }E\subseteq I\text{ and }E\in \mathcal{M}    
                    \]
                    Now let $x<y$, we see that $f(x)\leq f(y)$ therefore $g(x) = x+f(x)<y+f(y) = g(y)$, so this function is $1$ to $1$. Therefore disjoint sets in $I$ have disjoint $g$-images. The countable additivity of $m$ shows that $\mu$ is a positive bounded measure on $\mathcal{M}$. Furthermore since $g$ satisfies $(b)$ wew see that $\mu\ll m$ so\[
                        d\mu = h~dm    
                    \]
                    for some $h\in L^1(m)$, by Radon-Nikodym.

                    \

                    If $E = [a,x]$, then $g(E) = [g(a),g(b)]$  we have\[
                        g(x) - g(a) = m(g(E)) = \mu(E) = \int_E h~dm = \int_a^x h(t)~dt    
                    \]

                    So \[
                        f(x) - f(a) = (g(x)-g(a)) - (x-a) = \int_a^x (h(t)-1)~dt     
                    \]

                    Thus $f'(x) = h(x)-1$ a.e. $[m]$, by Theorem~\ref{7.11}.
                    \item $(c)\Rightarrow (a)$ This is shown in the remark from definition~\ref*{7.17}
                \end{itemize}
            \end{proof}
    \end{theorem}

    \begin{theorem}
        Suppose $f\colon I\rightarrow \R$ is AC and $I = [a,b]$. Define\[
            F(x) = \sup \sum_{i=1}^N|f(t_i)-f(t_{i-1})| \text{ } (a\leq x\leq b)   
        \]
        where the supremum is taken over all $N$ and over all choices of $\{t_i\}$ such that \[
            a=t_0<t_1<\cdots<t_N=x    
        \]

        The functions $F$, $F+f$, $F-f$ are then nondecreasing and $AC$ on $I$.
        \begin{proof}
            If for $\{t_i\}$ with the above property and $x<y\leq b$ then \[
                F(y)\geq |f(y)-f(x)|+\sum_{i=1}^N |f(t_i) - f(t_{i-1})|    
            \]
            So $F(y)\geq |f(y)-f(x)|+F(x)$. In particular \[
                F(y) \geq f(y)-f(x)+F(x)\text{ and }F(y)\geq f(x)-f(y)+F(x)    
            \]
            So $F,F+f,F-f$. Now we only need to show that $F$ is AC on $I$ since the sum of two AC functions is AC.

            If $(\alpha,\beta)\subseteq I$ then \[
                F(\beta)-F(\alpha) = \sum_{1}^n |f(t_i)-f(t_{i-1})|    
            \]

            Note that $\sum (t_i-t_{i-1}) = \beta - \alpha$.

            \

            Now let $\epsilon>0$ and associate $\delta>0$ to $f$ and $\epsilon$ like in~\ref{7.17}, choose disjoint segments $(\alpha_j,\beta_j)\subseteq I$ with $\sum(\beta_j-\alpha_j)<0$, it follows that\[
                \sum_j (F(\beta_j)-F(\alpha_j))\leq \epsilon    
            \]
            Thus $F$ is AC on $I$
        \end{proof}
    \end{theorem}

    \begin{definition}\index{total variation}
        The function $F$ defined in the theorem above is called the \textbf{total variation function} of $f$. If $f$ is any (complex) function on $I$ (AC or not), and $F(b)<\infty$, then $f$ is said to have \textbf{bounded variation} on $I$ and $F(b)$ is the \textbf{total variation} of $f$ on $I$.
    \end{definition}

    We have reached our main objective:
    \begin{theorem}
        If $f$ is a complex function that is AC on $I$, then $f$ is differentiable at almost all points of $I$, $f'\in L^1(m)$ and \[
            f(x)-f(a) = \int_a^x f'(t)~dt     
        \]
        \begin{proof}
            We just need to prove this for real $f$. Let $F$ be its total variation function and define:\[
                f_1 = \frac{1}{2}(F+f) \text{ and }f_2 = \frac{1}{2}(F-f)    
            \]
            We apply theorem~\ref*{7.18} to $f_1$ and $f_2$, and since \[
                f=f_1-f_2    
            \]
            We get \begin{align*}
                f(x)-f(a) &= (f_1(x)-f_1(a)) - (f_2(x)-f_2(a))\\
                &= \int_a^x f_1'(t)~dt -\int_a^x f_2'(t)~dt\\
                &= \int_a^x (f_1'-f_2')(t)~dt  
            \end{align*}
            By a previous theorem we $f' = f_1'-f_2'$ a.e. and we get our desired result.
        \end{proof}
    \end{theorem}

    \begin{theorem}
        If $f\colon [a,b]\rightarrow \R$ is differentiable at \textit{every} point of $[a,b]$ and $f'\in L^1$ on $[a,b]$, then \[
            f(x)-f(a) = \int_a^x f'(t)~dt    
        \]
        \begin{proof}
            We just need to prove this for $x=b$. Fix $\epsilon>0$, by a theorem from chap 2 we know there exists a lower semicontinuous function $g$ on $[a,b]$ such that $g>f'$ and \[
                \int_a^b g(t)~dt<\int_a^b f'(t)~dt+\epsilon    
            \]
            For any $\eta>0$ we define \[
                F_\eta(x) = \int_a^x g(t)~dt - f(x)+f(a)+\eta(x-a)    
            \]
            We keep $\eta$ fixed. For each $x\in [a,b)$ there corresponds a $\delta_x>0$ such that\[
                g(t)>f'(x) \text{ and }\frac{f(t)-f(x)}{t-x}<f'(x)+\eta    
            \]
            For all $t\in (x,x+\delta_x)$. Since $g$ is lower semicontinuous and $g(x)>f'(x)$. For any such $t$ we therefore have\[
                F_\eta(t) - F_\eta(x) = \int_x^t g(s)~ds - [f(t)-f(x)]+\eta(t-x) > (t-x)f'(x)-(t-x)(f'(x)+\eta)+\eta(t-x) = 0    
            \]

            Since $F_\eta(a) = 0$ and $F_\eta$ is continuous there is a last point $x\in [a,b]$ at which $F_\eta(x) = 0$. If $x<b$, the preceding computation implies that $F_\eta(t)>0$ for $t\in (x,b]$. In any case, $F_\eta(b)\geq 0$. Since this holds for every $\eta>0$ we see that \[
                f(b)-f(a)\leq \int_a^b g(t)~dt< \int_a^b f'(t)~dt + \epsilon    
            \] 
            Since $\epsilon$ was arbitrary we conclude that \[f(b)-f(a)\leq \int_a^b f'(t)~dt\] Furthermore $-f$ also satisfies the hypothesis of the theorem so the same inequality holds with $-f$ in the place of $f$, and these two inequalities give us the desired result.
        \end{proof}
    \end{theorem}
    \subsection{Differentiable Transformations}

    \printindex
\end{document}