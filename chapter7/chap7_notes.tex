\input{../template.tex}
\usepackage{makeidx}
\makeindex

\begin{document}
%\begin{theindex}
 %   \item Derivatives of Measures
  %  \subitem symmetric derivative~\ref{sym-der}
   % \subitem maximal function\ref{max-fun}
%\end{theindex}
\section{Differentiation}
\subsection{Derivatives of Measures}
\begin{theorem}
    Suppose $\mu$ is a complex Borel measure on $\R^1$ and \begin{equation}
        f(x) = \mu((-\infty, x)) \text{ for }x\in \R^1
    \end{equation}

    If $x\in \R^1$ and $A$ is a complex number, TFAE\begin{enumerate}[label = (\alph*)]
        \item $f$ is differentiable at $x$ and $f'(x) = A$.
        \item For all $\epsilon>0$, there exists $\delta>0$ such that\begin{equation}
            |\frac{\mu(I)}{m(I)}-A|<\epsilon
        \end{equation}

        for every open segment $I$ that contains $x$ and whose length is less than $\delta$. Note $m$ is the Lebesgue measure on $\R^1$.
    \end{enumerate}

    \begin{proof}
        $(a)\Rightarrow (b)$         
        Since $f'(x) = A$, we have, for all $\epsilon>0$ there is a $\delta>0$ such that for $(t,x)$ with $|t-x|<\delta$:\begin{align*}
            |\frac{f(t)-f(x)}{t-x} - f'(x)| &= |\frac{\mu([t,x))}{t-x} - A| = |\frac{\mu([t,x))}{m([t,x))} - A| <\epsilon \tag{$\dagger$}
        \end{align*}

        So for $I = (a,b)$ is any open interval containg $x$, of length less than $\delta$. Now let $\{t_n\}$ be such that $a<\ldots<t_n<t_{n-1}<\ldots < t_1$.
    \end{proof}
\end{theorem}

\begin{definition}
    Let us fix a dimension $k$, denote the open ball with center $x\in \R^k$ and radius $r>0$ by \[B(x,r) = \{y\in \R^k \colon |y-x|<r\}\]
    We associate to any Borel measure $\mu$ on $\R^k$ the quotients:\[({Q_r}\mu)(x) = \frac{\mu(B(x,r))}{m(B(x,r))}\]
    Where $m$ is the Lebesgue measure on $R^k$.

    \

    We define the \textbf{symmetric derivative}\index{symmetric derivative} to be \[(D\mu)(x) = \lim_{r\rightarrow 0}({Q_r}\mu)(x)\]
\end{definition}

\begin{definition}
    Using the same notation as above we define the \textbf{maximal function}\index{maximal function} $M\mu$, for $\mu\geq 0$, to be defined by \[(M\mu)(x) = \sup_{0<r<\infty}(Q_r\mu)(x)\]
\begin{remark}
    The maximal function of a complex Borel measure $\mu$ is, by definition, its total variation $|\mu|$.
\end{remark}
\end{definition}
\begin{proposition}
    The functions $M\mu\colon R^k\rightarrow [0,\infty]$ are lower semicontinuous, hence measurable.
    \begin{proof}
        Assume $\mu\geq 0$, and let $\lambda>0$ and $E = \{M\mu > \lambda\}$. Fix $x\in E$. Then there is an $r>0$ such that:\[\mu(B(x,r)) = tm(B(x,r)) \text{ for some }t>\lambda\]

        Indeed since $\sup_{0<r<\infty}\frac{\mu(B(x,r))}{m(B(x,r))} >\lambda$. So for some $r$, we have $\frac{\mu(B(x,r))}{m(B(x,r))} >\lambda$. Letting $t = \frac{\mu(B(x,r))}{m(B(x,r))}$ gives us the desired result. 
   
        Furthermore there is a $\delta>0$ such that:\[(r+\delta)^k<\frac{{r^k}t}{\lambda}\]
    If $|y-x|<\delta$, then $B(y,r+\delta) \supseteq B(x,r)$. Therefore
    \[
        \mu(B(y,r+\delta)) \geq \mu(B(x,r)) = tm(B(x,r)) = t[\frac{r}{{(r+\delta)}^k}m(B(y,r+\delta)) > \lambda m(B(y,r+\delta))]   
    \]
    Thus $B(x,\delta)\subseteq E$. So $E$ is open.
    \end{proof}
\end{proposition}

\begin{lemma}\label{7.3}
    If $W$ is the union of a finite collection of balls $B(x_i,r_i)$, with $i\leq i \leq N$. Then there is a set $S\subseteq \{1,\ldots,N\}$ so that:\begin{enumerate}[label = (\alph*)]
        \item the balls $B(x_i,r_i)$ with $i\in S$ are disjoint,
        \item \[W\subseteq \bigcup_{i\in S}B(x_i,3r_i),\]
        \item \[m(W)\leq 3^k\sum_{i\in S}m(B(x_i,r)i).\]
    \end{enumerate}

    \begin{proof}
        Order the balls $B_i = B(x_i,r_i)$ such that $r_1\geq r_2\geq \cdots \geq r_N$. Put $i_1 = 1$, discard all the $B_j$ that intersect with $B_{i_1}$. Let $B_{i_2}$ the first of our remaining balls, and discard all $B_j$ with $j>i_2$ that intersect $B_{i_2}$, and let $B_{i_3}$ be the first of the remaining ones, etc\dots

        This process stops after a finite number of steps, since we only have a finite collection of balls, and we let $S = \{i_1,i_2,\ldots\}$.
    $(a)$ holds by definition and $(c)$ follows from $(b)$ since $m(B(x_i,3r_i)) = {3^k}m(B(x_i,r_i))$.

    So we just need to show $(b)$. But notice for every discarded $B_j$, $B_j\cap B_i\neq \emptyset$ for some $i\in S$, where $r_i>r_j$. Assume that $X\in B_j\cap B_i$. We see that for all $x\in B_j$ we have:\begin{align*}
        |x-x_i| &\leq |x-X| + |X-x_i|\\
                &\leq |x-x_j| + |x_j-X| + |X-x_i|\\
                &< r_j + r_j + r_i \text{ since }x,X\in B_j\text{ and }X\in B_i\\
                &< 3r_i \text{ since }r_j\leq r_i
    \end{align*}
    So we see that $B_j\subseteq B(x_i,3r_i)$. This gives us $(b)$.
    \end{proof}
\end{lemma}

\paragraph*{The maximal theorem}\begin{theorem}\label{maximal}
    If $\mu$ is a complex Borel measure on $\R^k$ and $\lambda$ is a positive number, then\[m\{M\mu > \lambda\}\leq 3^k\lambda^{-1}||\mu|| \tag{i} \]
    Here $||\mu|| = |\mu|(\R^k)$ and $m\{M\mu > \lambda\}$ is an abbreviation of $m(\{x\in \R^k\colon (M\mu)(x) > \lambda\})$
\begin{proof}
    Fix $\mu$ and $\lambda$. Let $K$ be a compact subset of the open set $\{M\mu >\lambda\}$. Each $x\in K$ is the center of an open ball $B$ for which 
    \[|\mu|(B)>\lambda m(B)\]

    Some finite collection of these $B$'s covers $K$ and Lemma~\ref{7.3} tells us there is a disjoint subcollection $\{B_1,\ldots,B_n\}$ such that:\begin{equation*}
        m(K)\leq 3^k\sum_{1}^n m(B_i)\leq 3^k\lambda^{-1}\sum_{1}^n|\mu|(B_i) \leq 3^l\lambda^{-1}||\mu||
    \end{equation*}
    The disjointess of the $B_i$'s was used in the last inequality. So (i) follows by taking the supremum over all compact $K\subseteq \{M\mu > \lambda\}$.
\end{proof}
\end{theorem}
\paragraph*{Weak $L^1$}
If $f\in L^1(\R^k)$ and $\lambda >0$, then \[m\{|f|>\lambda\}\leq \lambda^{-1}||f||_1\]
because, if we let $E = \{|f|>\lambda\}$, we have:\begin{equation*}
    \lambda m(E)\leq \int_R |f|dm\leq \int_{\R^k}|f| dm = ||f||_1
\end{equation*}
\begin{definition}
    Any measurable function $f$ for which:\[\lambda m\{|f|>\lambda\}\]
    is a bounnded funtion of $\lambda$ on $(0,\infty)$ is said to belong to \textbf{weak}\index{weak} $L^1$\index{weak $L^1$}
 \end{definition}
So from above we see that the weak $L^1$ contains $L^1$. But it is also larger since for example if we let $f = \frac{1}{x}$ on $(0,1)$, then for any $\lambda>0$, we have
\[
    \frac{1}{x}>\lambda\iff x<\frac{1}{\lambda}    
\]

So we have $\lambda\cdot m\{|f|>\lambda\} \leq \lambda\cdot m(0,\frac{1}{\lambda}) = 1<\infty$. So $\frac{1}{x}$ is weak $L^1$.
\begin{definition}
    We associate to each $f\in L^1(\R^k)$ its \textbf{maximal function}\index{maximal function} $Mf\colon \R^k\rightarrow [0,\infty]$ by setting\begin{equation*}
        (Mf)(x) = \sup_{0<r<\infty}\frac{1}{m(B_r)}\int_{B(x,r)}|f|~dm
    \end{equation*}
\end{definition}
If we identify $f$ with the measure $\mu$ given by $d\mu = f~dm$, we see that this defintion agrees with the previously defined $M\mu$. So theorem~\ref{maximal} states that the ``maximal operator'' $M$ sends $L^1$ to weak $L^1$, witha bound (namely $3^k$)
that depends only on the space $\R^k$, i.e: For every $f\in L^1(\R^k)$ and every $\lambda>0$\begin{equation*}
    m\{Mf>\lambda\} \leq 3^k\lambda^{-1}||f||_1
\end{equation*}

\paragraph*{Lebesgue points} 
\begin{definition}
    If $f\in L^1(\R^k)$, any $x\in \R^k$ for which it is true that\[\lim_{r\rightarrow 0}\frac{1}{m(B_r)}\int_{B(x,r)} |f(y)-f(x)|~dm(y) = 0\]
    is called a \textbf{Lebesgue point}\index{Lebesgue point} of $f$.
\end{definition}
For example this equation holds if $f$ is continuous at the point $x$. More generally this equation holds, if the averages of $|f-f(x)|$ are not too small on the balls centered at $x$, i.e. The Lebesgue points of $f$ are thepoints where $f$ doesn't oscillate too much.

\begin{theorem}
    If $f\in L^1(R^k)$, then almost every $x\in \R^k$ is a Lebesgue point of $f$.
\end{theorem}

\printindex
\end{document}